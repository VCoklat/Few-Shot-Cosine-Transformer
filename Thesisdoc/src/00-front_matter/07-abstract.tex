%
% Halaman Abstract
%
% @author  Andreas Febrian
% @version 1.00
%

\chapter*{Abstract}

\vspace*{0.2cm}
{
	\setlength{\parindent}{0pt}
	
	\begin{tabular}{@{}l l p{10cm}}
		Name&: & \penulis \\
		Study Program&: & Computer Science \\
		Title&: & \judulInggris \\
		Supervisor 1&: & \pembimbing \\
        Supervisor 2&: & \pembimbin \\
	\end{tabular}

	\bigskip
	\bigskip

	Skin diseases represent a major public health issue, exacerbated by unequal access to dermatology services and lack of high-quality annotated data. Classical few-shot learning (FSL) methods use static regularization parameters that fail to adapt to varying episode difficulty. This research proposes Dynamic VIC Few-Shot Learning, integrating Variance-Invariance-Covariance (VIC) statistical regularization with dynamic weighting through an Episode-Adaptive Lambda Predictor. The architecture employs a Lightweight Cosine Transformer and SE-Enhanced Conv4 backbone optimized for computational efficiency. Evaluation on HAM10000 dataset demonstrates accuracy up to \textbf{77.44\%} with macro-F1 score of \textbf{0.7744} on Conv4 2-way 5-shot configuration---a \textbf{+20.52\%} improvement over baseline. The macro-F1 improvement indicates enhanced detection of clinically critical minority classes in extremely imbalanced datasets. With up to 90\% parameter reduction and high efficiency, this method shows potential as a skin disease classification solution for resource-limited primary healthcare facilities.

	\bigskip

	Key words:\\
	few-shot learning, skin disease classification, Dynamic VIC, Cosine Transformer, HAM10000
}

\newpage