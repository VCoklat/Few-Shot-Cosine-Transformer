%
% Halaman Abstrak
%
% @author  Andreas Febrian
% @version 1.00
%

\chapter*{Abstrak}

\vspace*{0.2cm}
{
	\setlength{\parindent}{0pt}
	
	\begin{tabular}{@{}l l p{10cm}}
		Nama&: & \penulis \\
		Program Studi&: & \program \\
		Judul&: & \judul \\
		Pembimbing 1&: & \pembimbing \\
        Pembimbing 2&: & \pembimbin \\
	\end{tabular}

	\bigskip
	\bigskip

	Penyakit kulit merupakan masalah kesehatan utama dengan distribusi layanan dermatologi yang timpang dan keterbatasan data berlabel. Metode \textit{few-shot learning} (FSL) klasik memiliki parameter regularisasi statis yang tidak adaptif terhadap variasi kesulitan antar episode. Penelitian ini mengusulkan \textit{Dynamic VIC Few-Shot Learning} yang mengintegrasikan regularisasi statistik \textit{Variance-Invariance-Covariance} (VIC) dengan mekanisme pembobotan dinamis melalui \textit{Episode-Adaptive Lambda Predictor}. Arsitektur menggunakan \textit{Lightweight Cosine Transformer} dan \textit{SE-Enhanced Conv4 backbone} yang dioptimalkan untuk efisiensi komputasi. Evaluasi pada dataset HAM10000 menunjukkan peningkatan macro-F1 score hingga \textbf{0,7744} pada konfigurasi Conv4 2-way 5-shot---peningkatan \textbf{+20,52\%} dari baseline. Peningkatan macro-F1 mengindikasikan kemampuan model mendeteksi kelas minoritas yang kritis secara klinis pada dataset dengan ketidakseimbangan ekstrem. Dengan pengurangan parameter hingga 90\% dan efisiensi yang tinggi, metode ini berpotensi sebagai solusi klasifikasi penyakit kulit untuk fasilitas kesehatan primer dengan sumber daya terbatas.

	\bigskip

	Kata kunci:\\
	few-shot learning, klasifikasi penyakit kulit, Dynamic VIC, Cosine Transformer, HAM10000
}

\newpage