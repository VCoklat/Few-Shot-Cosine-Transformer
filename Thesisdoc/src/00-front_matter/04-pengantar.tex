%
% Halaman Kata Pengantar
%

\chapter*{Kata Pengantar}
% \addcontentsline{toc}{chapter}{Kata Pengantar}

Puji syukur penulis panjatkan kepada Tuhan Yang Maha Esa atas segala rahmat dan karunia-Nya sehingga penulis dapat menyelesaikan tesis yang berjudul ``\judul'' ini dengan baik. Tesis ini disusun sebagai salah satu syarat untuk memperoleh gelar \gelar\ pada Program Studi \program, Fakultas \fakultas, Universitas Indonesia.

Penulis menyadari bahwa penyelesaian tesis ini tidak terlepas dari dukungan, bimbingan, dan bantuan dari berbagai pihak. Oleh karena itu, pada kesempatan ini penulis ingin menyampaikan ucapan terima kasih yang sebesar-besarnya kepada:

\begin{enumerate}
    \item \pembimbing\ selaku Pembimbing I yang telah memberikan bimbingan, arahan, ilmu pengetahuan, serta motivasi yang sangat berharga selama proses penyusunan tesis ini.
    
    \item \pembimbin\ selaku Pembimbing II yang telah memberikan masukan, saran, dan bimbingan yang sangat membantu dalam penyempurnaan tesis ini.
    
    \item Seluruh dosen, penguji dan staf Program Studi \program\ Fakultas \fakultas\ Universitas Indonesia yang telah memberikan ilmu dan bantuan selama penulis menempuh pendidikan.
    
    \item Kedua orang tua, adik saya Marla Marlena keluarga besar tercinta, serta pasangan saya Laurensia Jeany yang senantiasa memberikan doa, dukungan moral, menemani setiap tahap dan memberikan motivasi yang tidak pernah putus kepada penulis.

    \item Para asisten lab, dari mas luthfi, mas yohanes, mas hannan, mas yogiek, mas fadhil, mas awan dan para asisten lainnya yang telah mau membantu, menjadi teman diskusi dan saya repotkan
    
    \item Rekan-rekan mahasiswa Program Magister \program\ terutama mahasiswa bimbingan Lab IROS 1231 yang telah berbagi pengalaman, pengetahuan, dan semangat selama menjalani perkuliahan.
    
    \item Semua pihak yang tidak dapat penulis sebutkan satu per satu yang telah membantu dalam penyelesaian tesis ini.
\end{enumerate}

Penulis menyadari bahwa tesis ini masih jauh dari sempurna.  Oleh karena itu, kritik dan saran yang membangun sangat penulis harapkan demi perbaikan di masa yang akan datang.  Semoga tesis ini dapat memberikan manfaat dan kontribusi bagi perkembangan ilmu pengetahuan, khususnya dalam bidang \textit{few-shot learning} dan klasifikasi penyakit kulit.

\vspace{1.5cm}
\begin{flushright}
    Depok, \bulan\ \tahun\\[1.5cm]
    \penulis
\end{flushright}

\newpage